% -*-latex-*-
% 
% For questions, comments, concerns or complaints:
% thesis@mit.edu
% 
%
% $Log: cover.tex,v $
% Revision 1.8  2008/05/13 15:02:15  jdreed
% Degree month is June, not May.  Added note about prevdegrees.
% Arthur Smith's title updated
%
% Revision 1.7  2001/02/08 18:53:16  boojum
% changed some \newpages to \cleardoublepages
%
% Revision 1.6  1999/10/21 14:49:31  boojum
% changed comment referring to documentstyle
%
% Revision 1.5  1999/10/21 14:39:04  boojum
% *** empty log message ***
%
% Revision 1.4  1997/04/18  17:54:10  othomas
% added page numbers on abstract and cover, and made 1 abstract
% page the default rather than 2.  (anne hunter tells me this
% is the new institute standard.)
%
% Revision 1.4  1997/04/18  17:54:10  othomas
% added page numbers on abstract and cover, and made 1 abstract
% page the default rather than 2.  (anne hunter tells me this
% is the new institute standard.)
%
% Revision 1.3  93/05/17  17:06:29  starflt
% Added acknowledgements section (suggested by tompalka)
% 
% Revision 1.2  92/04/22  13:13:13  epeisach
% Fixes for 1991 course 6 requirements
% Phrase "and to grant others the right to do so" has been added to 
% permission clause
% Second copy of abstract is not counted as separate pages so numbering works
% out
% 
% Revision 1.1  92/04/22  13:08:20  epeisach

% NOTE:
% These templates make an effort to conform to the MIT Thesis specifications,
% however the specifications can change.  We recommend that you verify the
% layout of your title page with your thesis advisor and/or the MIT 
% Libraries before printing your final copy.
\newgeometry{left=3.5cm,bottom=3.5cm}

\title{LearnAir: toward Intelligent, Personal Air Quality Monitoring}

\author{David B. Ramsay}
% If you wish to list your previous degrees on the cover page, use the 
% previous degrees command:
\prevdegrees{BSEE and BA, Case Western Reserve University (2010)}
% You can use the \\ command to list multiple previous degrees
\department{Program in Media Arts and Sciences, School of Architecture and Planning}

% If the thesis is for two degrees simultaneously, list them both
% separated by \and like this:
% \degree{Doctor of Philosophy \and Master of Science}
\degree{Master of Science}

% As of the 2007-08 academic year, valid degree months are September, 
% February, or June.  The default is June.
\degreemonth{September}
\degreeyear{2016}
\thesisdate{August 6, 2016}

%% By default, the thesis will be copyrighted to MIT.  If you need to copyright
%% the thesis to yourself, just specify the `vi' documentclass option.  If for
%% some reason you want to exactly specify the copyright notice text, you can
%% use the \copyrightnoticetext command.  
%\copyrightnoticetext{\copyright IBM, 1990.  Do not open till Xmas.}

% If there is more than one supervisor, use the \supervisor command
% once for each.
\supervisor{Joseph A. Paradiso}{Professor of Media Arts and Sciences}

% This is the department committee chairman, not the thesis committee
% chairman.  You should replace this with your Department's Committee
% Chairman.
\chairman{Pattie Maes}{Academic Head\\Program in Media Arts and Sciences}

% Make the titlepage based on the above information.  If you need
% something special and can't use the standard form, you can specify
% the exact text of the titlepage yourself.  Put it in a titlepage
% environment and leave blank lines where you want vertical space.
% The spaces will be adjusted to fill the entire page.  The dotted
% lines for the signatures are made with the \signature command.
\maketitle

% The abstractpage environment sets up everything on the page except
% the text itself.  The title and other header material are put at the
% top of the page, and the supervisors are listed at the bottom.  A
% new page is begun both before and after.  Of course, an abstract may
% be more than one page itself.  If you need more control over the
% format of the page, you can use the abstract environment, which puts
% the word "Abstract" at the beginning and single spaces its text.

%% You can either \input (*not* \include) your abstract file, or you can put
%% the text of the abstract directly between the \begin{abstractpage} and
%% \end{abstractpage} commands.

% First copy: start a new page, and save the page number.
\cleardoublepage
% Uncomment the next line if you do NOT want a page number on your
% abstract and acknowledgments pages.
\pagestyle{empty}
\setcounter{savepage}{\thepage}

\newgeometry{bottom=0.1cm}
\begin{abstractpage}
%\fancypagestyle{customstyle}{%
%\fancyhf{}%
%	\fancyhead[LE]{\thepage\quad\smallcaps{\newlinetospace{}}}% 
%	\fancyhead[RO]{\smallcaps{\newlinetospace{}}\quad\thepage}%
}

% $Log: abstract.tex,v $
% Revision 1.1  93/05/14  14:56:25  starflt
% Initial revision
% 
% Revision 1.1  90/05/04  10:41:01  lwvanels
% Initial revision
% 
%
%% The text of your abstract and nothing else (other than comments) goes here.
%% It will be single-spaced and the rest of the text that is supposed to go on
%% the abstract page will be generated by the abstractpage environment.  This
%% file should be \input (not \include 'd) from cover.tex.


Air pollution is responsible for 1/8 of deaths around the world, and has become a huge concern for urban citizens the world over.  Studies have conclusively proven that the status quo of sparse, fixed sensors cannot accurately capture personal exposure levels of nearby populations.  Especially in urban landscapes, pollutant concentrations can swing wildly over just a few seconds or a few meters.  Unfortunately, the portable monitors  that are capable of accurately measuring these pollutants cost thousands of dollars.
  
That hasn't stopped a deluge of cheap, portable consumer devices from entering the market.  These solutions frequently claim better accuracy, but universally fail under real-world validation.  Instead of competing to \textit{build} a more accurate sensor, we take the approach of trying to \textit{predict} when we can trust the cheap sensor we have, based on ambient conditions and measurements.

Well-designed, sub-\$100 sensors are just recently gaining the reputation of their higher quality peers.  While their fundamental operation is sound, these cheaper alternatives cannot incorporate the costly, industry standard techniques for mitigating issues like cross sensitivity, dynamic airflow, or high humidity.  Fortunately, if the core principles of the device are sound, machine learning techniques can be applied to analyze and predict systemic measurement failure based on a handful of related indicators.  In this thesis, we tested and demonstrated the potential for logistic regression machine learning techniques to predict and classify sensor measurements as `correct' or `incorrect' with very high reliability.  These techniques are also useful for quantifying sensor precision, as well as the minimum training set duration for reliable prediction across seasons. 

After demonstrating the value of this approach, we went on to design a scalable database solution using a technology know as ChainAPI.  The tools developed in this framework allow for automatic learning algorithms to crawl through the database structure pulling and learning on all of the most recent, up-to-date data, as well as populate the database with processed data for other crawling scripts to interact with.  This backend has interesting implications for air quality data storage, interaction, and exchange.

Finally, we built a portable, BLE enabled air quality device that connects to ChainAPI through a mobile phone app, and takes advantage of the machine learning algorithms running in its backend.  This device improves the trust and reliability of sensor data, compared with similarly cost systems. 

The LearnAir device empowers individuals to trust their personal air quality data, and enable a dialog about sensor reliability within the citizen sensing community.  Its novel, backend design promotes new ways of collaborating with air quality data, interacting with dynamic datasets, and automatic characterization of new devices against reliable references.  Finally, the logistic regression techniques increase the usefulness of cheap, distributed sensor data-- creating a trusted way for researchers to collaborate with citizen data from around the world.
\end{abstractpage}
% Additional copy: start a new page, and reset the page number.  This way,
% the second copy of the abstract is not counted as separate pages.
% Uncomment the next 6 lines if you need two copies of the abstract
% page.
% \setcounter{page}{\thesavepage}
% \begin{abstractpage}
% % $Log: abstract.tex,v $
% Revision 1.1  93/05/14  14:56:25  starflt
% Initial revision
% 
% Revision 1.1  90/05/04  10:41:01  lwvanels
% Initial revision
% 
%
%% The text of your abstract and nothing else (other than comments) goes here.
%% It will be single-spaced and the rest of the text that is supposed to go on
%% the abstract page will be generated by the abstractpage environment.  This
%% file should be \input (not \include 'd) from cover.tex.

In this thesis, I describe why I am worthy of an MIT degree.  Lorem ipsum dolor sit amet, consectetur adipiscing elit. Aenean quis dolor bibendum, lobortis mauris a, sollicitudin lacus. Vivamus sollicitudin orci sed convallis faucibus. Morbi tempor augue vel nunc mollis euismod. Fusce varius fermentum dui, vel ultrices massa fermentum a. Pellentesque ac ipsum et libero cursus posuere. Aliquam tincidunt sapien ut ultrices dignissim. Cras tortor leo, pulvinar sagittis lacus et, convallis consectetur quam. Suspendisse potenti. Etiam convallis velit felis, eu rutrum ligula dictum sit amet.


% \end{abstractpage}

\cleardoublepage

% READER PAGE
\reader{Steven Hamburg}{Chief Scientist}{Environmental Defense Fund}

\readerpage

\clearpage

\reader{Ethan Zuckerman}{Associate Professor of the Practice}{Program in Media Arts and Sciences}

\readerpage

\cleardoublepage
\newgeometry{bottom=5cm}

\section*{Acknowledgments}

Many selfless people advised, guided, and supported me during this thesis.  I'm humbled and grateful to count them all as close friends and collaborators.  I couldn't have done this without them.

First I'd like to thank my advisor, \textbf{Joe Paradiso}.  He not only laid the groundwork for this thesis with his depth of expertise and strong connections; he has been an incredibly thoughtful, intelligent, and warm advisor.  I will always be grateful for the chance he took on me, to join his lab and represent his vision in the world.  I hope I might one day be half as sharp, capable, and kind.

The Enivornmental Defense Fund- particularly my reader \textbf{Dr. Steven Hamburg} and his collaborator \textbf{Millie Chu Baird}- have been instrumental in shaping the direction of this work, and providing great insight and invaluable community connections.  They supported my Media Lab appointment and this work; it would not exist without them.  I'm forever grateful not just for their funding, but for their active engagement from the very beginning.

\textbf{Ethan Zuckerman} has been a true mentor to me since this process began.  I've never met a more helpful, brilliant, encouraging, and eloquent professor.  He has been integral in shaping this project, but the most valuable things I've gleaned from him fall far beyond the bounds of this thesis and will stay with me well past its completion.  

Ethan's group shares his spirit, and I'm indebted to the entire Civic Media Team.  \textbf{Emilie Reiser} has been a brilliant collaborator and a wonderful friend throughout this process, investing countless hours helping and challenging me.  \textbf{Don Blair} has also been incredibly warm, thoughtful, and constructive over many hours of conversation.  The extended Civic family has been very generous with their time, and I'd like to thank all of them, particularly \textbf{Dave Mackintosh}, \textbf{Xiuli Wang}, and \textbf{Colin McCormick}.

I've relied on several experts to shape and inform my thinking about this project.  In particular, I'm indebted to Safecast's \textbf{Sean Bonner} and \textbf{Pieter Franken} for their insight, which launched me into this project with a strong foundation.  As I've continued, \textbf{Dr. Jesse Kroll} and \textbf{David Hagan} from MIT's Civil and Environmental Engineering Department have been very unselfish with their time and expertise.  Their technical mastery of the field is truly inspiring.     

I'm completely beholden to MassDEP- particularly \textbf{John Lane} and \textbf{Tom McGrath}- for allowing me 24/7 access to the EPA measurement site.  This thesis wouldn't exist without their flexibility, and they've been delightful, responsive, and accomodating collaborators. 

I'd also like to acknowledge my Responsive Environments family- particularly \textbf{Spencer Russell}, whose CHAIN work forms the basis for much of my contribution here (and who spent a tremendous amount of time helping me understand how to use it), and \textbf{Brian Mayton} for his technical insight and advice throughout the process.  A big thanks goes to \textbf{Nan, Evan, Juliana, Asaf, Artem, Jie, Donald, Vasant, and Gershon} for useful, fun, and inspiring conversations along the way.  It's a pleasure to work with people I admire so much.

\textbf{Amna, Keira, and Linda}- you three have kept me on track and been incredibly flexible and kind throughout the last two years.  Thank you for the support, the smiles, and the gentle reminders.  To my Boston friends - \textbf{Kristy, Chetan, Will, Nate, and Dylan} - thanks for putting up with me and keeping me sane throughout this process.  

On a personal note, I'd like to thank the mentors that have invested in me, shaped me into who I am, and continue to challenge, guide, and inspire me.  \textbf{Dan Gauger}, \textbf{Neal Lackritz}, \textbf{Ted Burke}, and \textbf{bunnie} - I can't overstate the impact you each have had on my life- as technical mentors certainly, but more importantly as confidants and role models.  You inspire me to think creatively and make a difference through ambitious, high-quality work.  You motivate me to live a more balanced life and approach the world with kindness and gratitude.  You challenge me to re-examine my priorities, my goals, and my philosophies through your example.  You've each made an indelible impact on my life, and I will continue to strive to follow in your example.   

Most of all, I'm indebted to my wonderful and supportive family- my parents \textbf{Karen and Brad}, my sister \textbf{Tracy}, and the entire Benson/Ramsay clan.  Thank you for believing in me, pushing me, and guiding me throughout the last 29 years.  I admire you, I love you, and I owe you everything.



\restoregeometry

%%%%%%%%%%%%%%%%%%%%%%%%%%%%%%%%%%%%%%%%%%%%%%%%%%%%%%%%%%%%%%%%%%%%%%
% -*-latex-*-
