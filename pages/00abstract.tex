% $Log: abstract.tex,v $
% Revision 1.1  93/05/14  14:56:25  starflt
% Initial revision
% 
% Revision 1.1  90/05/04  10:41:01  lwvanels
% Initial revision
% 
%
%% The text of your abstract and nothing else (other than comments) goes here.
%% It will be single-spaced and the rest of the text that is supposed to go on
%% the abstract page will be generated by the abstractpage environment.  This
%% file should be \input (not \include 'd) from cover.tex.


Air pollution is responsible for 1/8 of deaths around the world, and has become a huge concern for urban citizens the world over.  Studies have conclusively proven that the status quo of sparse, fixed sensors cannot accurately capture personal exposure levels of nearby populations.  Especially in urban landscapes, pollutant concentrations can swing wildly over just a few seconds or a few meters.  Unfortunately, the portable monitors  that are capable of accurately measuring these pollutants cost thousands of dollars.
  
That hasn't stopped a deluge of cheap, portable consumer devices from entering the market.  These solutions frequently claim better accuracy, but universally fail under real-world validation.  Instead of competing to \textit{build} a more accurate sensor, we take the approach of trying to \textit{predict} when we can trust the cheap sensor we have, based on ambient conditions and measurements.

Well-designed, sub-\$100 sensors are just recently gaining the reputation of their higher quality peers.  While their fundamental operation is sound, these cheaper alternatives cannot incorporate the costly, industry standard techniques for mitigating issues like cross sensitivity, dynamic airflow, or high humidity.  Fortunately, if the core principles of the device are sound, machine learning techniques can be applied to analyze and predict systemic measurement failure based on a handful of related indicators.  In this thesis, we tested and demonstrated the potential for logistic regression machine learning techniques to predict and classify sensor measurements as `correct' or `incorrect' with very high reliability.  These techniques are also useful for quantifying sensor precision, as well as the minimum training set duration for reliable prediction across seasons. 

After demonstrating the value of this approach, we went on to design a scalable database solution using a technology know as ChainAPI.  The tools developed in this framework allow for automatic learning algorithms to crawl through the database structure pulling and learning on all of the most recent, up-to-date data, as well as populate the database with processed data for other crawling scripts to interact with.  This backend has interesting implications for air quality data storage, interaction, and exchange.

Finally, we built a portable, BLE enabled air quality device that connects to ChainAPI through a mobile phone app, and takes advantage of the machine learning algorithms running in its backend.  This device improves the trust and reliability of sensor data, compared with similarly cost systems. 

The LearnAir device empowers individuals to trust their personal air quality data, and enable a dialog about sensor reliability within the citizen sensing community.  Its novel, backend design promotes new ways of collaborating with air quality data, interacting with dynamic datasets, and automatic characterization of new devices against reliable references.  Finally, the logistic regression techniques increase the usefulness of cheap, distributed sensor data-- creating a trusted way for researchers to collaborate with citizen data from around the world.