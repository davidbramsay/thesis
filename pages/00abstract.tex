% $Log: abstract.tex,v $
% Revision 1.1  93/05/14  14:56:25  starflt
% Initial revision
% 
% Revision 1.1  90/05/04  10:41:01  lwvanels
% Initial revision
% 
%
%% The text of your abstract and nothing else (other than comments) goes here.
%% It will be single-spaced and the rest of the text that is supposed to go on
%% the abstract page will be generated by the abstractpage environment.  This
%% file should be \input (not \include 'd) from cover.tex.


---This is the OLD Abstract from my proposal, and is not up-to-date.  It will be updated shortly---

Air pollution is responsible for 1/8 of deaths around the world.  Studies have conclusively proven that the status quo of sparse, fixed sensors cannot accurately capture personal exposure levels of nearby populations.  Especially in urban landscapes, pollutant concentrations can swing wildly over just a few seconds or a few meters.  
Increasingly, mobile solutions are being tested for mapping cities and measuring personal exposure.  Affordable and portable sensing technology has just reached the accuracy necessary for meaningful data collection.  Unfortunately, current systems are not robust at dealing with rapid pollution variation, dynamic airflow, or changes in temperature, humidity, and pressure.

For this work, we will explore novel techniques for addressing the shortcomings of the two types of sensors most commonly used for mobile sensing-- Particulate Sensing and Electrochemical Gas Sensing.  We will prototype several new designs in an attempt to mitigate issues with airflow, spatiotemporal pollutant variation, temperature, humidity, and pressure. We will test and characterize our designs (1) in controlled dynamics and airflow conditions, (2) in real urban settings, and (3) in comparison with higher quality reference sensors, with a particular focus on how they perform under various speeds of travel (walking, running, biking, and driving). Combined with a smartphone, we believe we can produce a reliable and affordable sensor platform.

Ultimately, we want to empower citizen groups to understand and combat air pollution in their community.  For people who are actively trying to mitigate their exposure to harmful pollutants, we want to help them understand their exposure, their risk, and whether their actions are working.  For communities where no data exists, we want to empower affordable, distributed, and verifiably accurate data collection that can serve as a basis for civic discourse.  We believe these are achievable goals with the current technology, as long as these designs are truly built and tested for the mobile context.
