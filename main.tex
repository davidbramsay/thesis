% -*- Mode:TeX -*-

%% IMPORTANT: The official thesis specifications are available at:
%%            http://libraries.mit.edu/archives/thesis-specs/
%%
%%            Please verify your thesis' formatting and copyright
%%            assignment before submission.  If you notice any
%%            discrepancies between these templates and the 
%%            MIT Libraries' specs, please let us know
%%            by e-mailing thesis@mit.edu

%% The documentclass options along with the pagestyle can be used to generate
%% a technical report, a draft copy, or a regular thesis.  You may need to
%% re-specify the pagestyle after you \include  cover.tex.  For more
%% information, see the first few lines of mitthesis.cls. 

%\documentclass[12pt,vi,twoside]{mitthesis}o
%%
%%  If you want your thesis copyright to you instead of MIT, use the
%%  ``vi'' option, as above.
%%
%\documentclass[12pt,twoside,leftblank]{mitthesis}
%%
%% If you want blank pages before new chapters to be labelled ``This
%% Page Intentionally Left Blank'', use the ``leftblank'' option, as
%% above. 

\documentclass[nobib]{tufte-book}
\usepackage{lgrind}
\usepackage{graphicx}
\usepackage{lipsum}
\usepackage{booktabs}
\usepackage{multirow}
\usepackage{moreverb}
\usepackage{alltt}
\usepackage{parskip}
\usepackage{url}
\usepackage{array}
\usepackage{pdfpages}
\usepackage{wrapfig}
\usepackage{geometry}
\setkeys{Gin}{width=\linewidth,totalheight=\textheight,keepaspectratio}

%% These have been added at the request of the MIT Libraries, because
%% some PDF conversions mess up the ligatures.  -LB, 1/22/2014
\usepackage{fancyvrb}

%%
% Prints argument within hanging parentheses (i.e., parentheses that take
% up no horizontal space).  Useful in tabular environments.
\newcommand{\hangp}[1]{\makebox[0pt][r]{(}#1\makebox[0pt][l]{)}}

%%
% Prints an asterisk that takes up no horizontal space.
% Useful in tabular environments.
\newcommand{\hangstar}{\makebox[0pt][l]{*}}

%%
% Prints a trailing space in a smart way.
\usepackage{xspace}


% custom page numbering
\fancypagestyle{customstyle}{%
\fancyhf{}%
	\fancyhead[LE]{\thepage\quad\smallcaps{\newlinetospace{}}}% 
	\fancyhead[RO]{\smallcaps{\newlinetospace{}}\quad\thepage}%
}

\usepackage[parfill]{parskip}

% remove paragraph indentation
\makeatletter
% Paragraph indentation and separation for normal text
\renewcommand{\@tufte@reset@par}{%
  \setlength{\RaggedRightParindent}{0.0pc}%
  \setlength{\JustifyingParindent}{0.0pc}%
  \setlength{\parindent}{0pc}%
  \setlength{\parskip}{\baselineskip}%
}
\@tufte@reset@par

% Paragraph indentation and separation for marginal text
\renewcommand{\@tufte@margin@par}{%
  \setlength{\RaggedRightParindent}{0.0pc}%
  \setlength{\JustifyingParindent}{0.0pc}%
  \setlength{\parindent}{0.0pc}%
  \setlength{\parskip}{10pt}%
}
\makeatother

%% This bit allows you to either specify onwly the files which you wish to
%% process, or `all' to process all files which you \include.
%% Krishna Sethuraman (1990).

% \setcounter{secnumdepth}{0}
% \includeonly{chap2}

\begin{document}

% -*-latex-*-
% 
% For questions, comments, concerns or complaints:
% thesis@mit.edu
% 
%
% $Log: cover.tex,v $
% Revision 1.8  2008/05/13 15:02:15  jdreed
% Degree month is June, not May.  Added note about prevdegrees.
% Arthur Smith's title updated
%
% Revision 1.7  2001/02/08 18:53:16  boojum
% changed some \newpages to \cleardoublepages
%
% Revision 1.6  1999/10/21 14:49:31  boojum
% changed comment referring to documentstyle
%
% Revision 1.5  1999/10/21 14:39:04  boojum
% *** empty log message ***
%
% Revision 1.4  1997/04/18  17:54:10  othomas
% added page numbers on abstract and cover, and made 1 abstract
% page the default rather than 2.  (anne hunter tells me this
% is the new institute standard.)
%
% Revision 1.4  1997/04/18  17:54:10  othomas
% added page numbers on abstract and cover, and made 1 abstract
% page the default rather than 2.  (anne hunter tells me this
% is the new institute standard.)
%
% Revision 1.3  93/05/17  17:06:29  starflt
% Added acknowledgements section (suggested by tompalka)
% 
% Revision 1.2  92/04/22  13:13:13  epeisach
% Fixes for 1991 course 6 requirements
% Phrase "and to grant others the right to do so" has been added to 
% permission clause
% Second copy of abstract is not counted as separate pages so numbering works
% out
% 
% Revision 1.1  92/04/22  13:08:20  epeisach

% NOTE:
% These templates make an effort to conform to the MIT Thesis specifications,
% however the specifications can change.  We recommend that you verify the
% layout of your title page with your thesis advisor and/or the MIT 
% Libraries before printing your final copy.
\newgeometry{left=3.5cm,bottom=0.1cm}

\title{My MIT Thesis}

\author{Student Name}
% If you wish to list your previous degrees on the cover page, use the 
% previous degrees command:
% \prevdegrees{A.A., Harvard University (1985)}
% You can use the \\ command to list multiple previous degrees
\department{Department of IHTFP}

% If the thesis is for two degrees simultaneously, list them both
% separated by \and like this:
% \degree{Doctor of Philosophy \and Master of Science}
\degree{Doctor of Philosophy in IHTFP}

% As of the 2007-08 academic year, valid degree months are September, 
% February, or June.  The default is June.
\degreemonth{June}
\degreeyear{2016}
\thesisdate{May 20, 2016}

%% By default, the thesis will be copyrighted to MIT.  If you need to copyright
%% the thesis to yourself, just specify the `vi' documentclass option.  If for
%% some reason you want to exactly specify the copyright notice text, you can
%% use the \copyrightnoticetext command.  
%\copyrightnoticetext{\copyright IBM, 1990.  Do not open till Xmas.}

% If there is more than one supervisor, use the \supervisor command
% once for each.
\supervisor{William Barton Rogers}{Professor}

% This is the department committee chairman, not the thesis committee
% chairman.  You should replace this with your Department's Committee
% Chairman.
\chairman{Susan Hockfield}{Chair(woman)}

% Make the titlepage based on the above information.  If you need
% something special and can't use the standard form, you can specify
% the exact text of the titlepage yourself.  Put it in a titlepage
% environment and leave blank lines where you want vertical space.
% The spaces will be adjusted to fill the entire page.  The dotted
% lines for the signatures are made with the \signature command.
\maketitle

% The abstractpage environment sets up everything on the page except
% the text itself.  The title and other header material are put at the
% top of the page, and the supervisors are listed at the bottom.  A
% new page is begun both before and after.  Of course, an abstract may
% be more than one page itself.  If you need more control over the
% format of the page, you can use the abstract environment, which puts
% the word "Abstract" at the beginning and single spaces its text.

%% You can either \input (*not* \include) your abstract file, or you can put
%% the text of the abstract directly between the \begin{abstractpage} and
%% \end{abstractpage} commands.

% First copy: start a new page, and save the page number.
\cleardoublepage
% Uncomment the next line if you do NOT want a page number on your
% abstract and acknowledgments pages.
% \pagestyle{empty}
\setcounter{savepage}{\thepage}
\begin{abstractpage}
% $Log: abstract.tex,v $
% Revision 1.1  93/05/14  14:56:25  starflt
% Initial revision
% 
% Revision 1.1  90/05/04  10:41:01  lwvanels
% Initial revision
% 
%
%% The text of your abstract and nothing else (other than comments) goes here.
%% It will be single-spaced and the rest of the text that is supposed to go on
%% the abstract page will be generated by the abstractpage environment.  This
%% file should be \input (not \include 'd) from cover.tex.

In this thesis, I describe why I am worthy of an MIT degree.  Lorem ipsum dolor sit amet, consectetur adipiscing elit. Aenean quis dolor bibendum, lobortis mauris a, sollicitudin lacus. Vivamus sollicitudin orci sed convallis faucibus. Morbi tempor augue vel nunc mollis euismod. Fusce varius fermentum dui, vel ultrices massa fermentum a. Pellentesque ac ipsum et libero cursus posuere. Aliquam tincidunt sapien ut ultrices dignissim. Cras tortor leo, pulvinar sagittis lacus et, convallis consectetur quam. Suspendisse potenti. Etiam convallis velit felis, eu rutrum ligula dictum sit amet.


\end{abstractpage}

% Additional copy: start a new page, and reset the page number.  This way,
% the second copy of the abstract is not counted as separate pages.
% Uncomment the next 6 lines if you need two copies of the abstract
% page.
% \setcounter{page}{\thesavepage}
% \begin{abstractpage}
% % $Log: abstract.tex,v $
% Revision 1.1  93/05/14  14:56:25  starflt
% Initial revision
% 
% Revision 1.1  90/05/04  10:41:01  lwvanels
% Initial revision
% 
%
%% The text of your abstract and nothing else (other than comments) goes here.
%% It will be single-spaced and the rest of the text that is supposed to go on
%% the abstract page will be generated by the abstractpage environment.  This
%% file should be \input (not \include 'd) from cover.tex.

In this thesis, I describe why I am worthy of an MIT degree.  Lorem ipsum dolor sit amet, consectetur adipiscing elit. Aenean quis dolor bibendum, lobortis mauris a, sollicitudin lacus. Vivamus sollicitudin orci sed convallis faucibus. Morbi tempor augue vel nunc mollis euismod. Fusce varius fermentum dui, vel ultrices massa fermentum a. Pellentesque ac ipsum et libero cursus posuere. Aliquam tincidunt sapien ut ultrices dignissim. Cras tortor leo, pulvinar sagittis lacus et, convallis consectetur quam. Suspendisse potenti. Etiam convallis velit felis, eu rutrum ligula dictum sit amet.


% \end{abstractpage}

\cleardoublepage

% READER PAGE
\reader{Reader 1 Name}{Reader 1 Title}{Reader 1 Affiliation}
\reader{Reader 2 Name}{Reader 2 Title}{Reader 2 Affiliation}

\readerpage

\cleardoublepage

\section*{Acknowledgments}

This is the acknowledgements section.  You should replace this with your
own acknowledgements.

\restoregeometry

%%%%%%%%%%%%%%%%%%%%%%%%%%%%%%%%%%%%%%%%%%%%%%%%%%%%%%%%%%%%%%%%%%%%%%
% -*-latex-*-

% Some departments (e.g. 5) require an additional signature page.  See
% signature.tex for more information and uncomment the following line if
% applicable.
% % -*- Mode:TeX -*-
%
% Some departments (e.g. Chemistry) require an additional cover page
% with signatures of the thesis committee.  Please check with your
% thesis advisor or other appropriate person to determine if such a 
% page is required for your thesis.  
%
% If you choose not to use the "titlepage" environment, a \newpage
% commands, and several \vspace{\fill} commands may be necessary to
% achieve the required spacing.  The \signature command is defined in
% the "mitthesis" class
%
% The following sample appears courtesy of Ben Kaduk <kaduk@mit.edu> and
% was used in his June 2012 doctoral thesis in Chemistry. 

\begin{titlepage}
\begin{large}
This doctoral thesis has been examined by a Committee of the Department
of Chemistry as follows:

\signature{Professor Jianshu Cao}{Chairman, Thesis Committee \\
   Professor of Chemistry}

\signature{Professor Troy Van Voorhis}{Thesis Supervisor \\
   Associate Professor of Chemistry}

\signature{Professor Robert W. Field}{Member, Thesis Committee \\
   Haslam and Dewey Professor of Chemistry}
\end{large}
\end{titlepage}


\pagestyle{customstyle}
  % -*- Mode:TeX -*-
%% This file simply contains the commands that actually generate the table of
%% contents and lists of figures and tables.  You can omit any or all of
%% these files by simply taking out the appropriate command.  For more
%% information on these files, see appendix C.3.3 of the LaTeX manual. 
\tableofcontents
\newpage
\listoffigures
\newpage
\listoftables


%% This is an example first chapter.  You should put chapter/appendix that you
%% write into a separate file, and add a line \include{yourfilename} to
%% main.tex, where `yourfilename.tex' is the name of the chapter/appendix file.
%% You can process specific files by typing their names in at the 
%% \files=
%% prompt when you run the file main.tex through LaTeX.
\chapter{1. Introduction}

\begin{quote}
\textit{In this chapter, I describe my introduction. Lorem ipsum dolor sit amet, consectetur adipiscing elit. Aenean quis dolor bibendum, lobortis mauris a, sollicitudin lacus.} \newline
\end{quote}

Lorem ipsum dolor sit amet, consectetur adipiscing elit. Aenean quis dolor bibendum, lobortis mauris a, sollicitudin lacus. Vivamus sollicitudin orci sed convallis faucibus. Morbi tempor augue vel nunc mollis euismod. Fusce varius fermentum dui, vel ultrices massa fermentum a. Pellentesque ac ipsum et libero cursus posuere. Aliquam tincidunt sapien ut ultrices dignissim. Cras tortor leo, pulvinar sagittis lacus et, convallis consectetur quam. Suspendisse potenti. Etiam convallis velit felis, eu rutrum ligula dictum sit amet (Figure \ref{fig:spin_margin}).

\marginnote{\textbf{Margin Note:} Check it out, here's a margin note.}

Sed nec suscipit ex. Ut quis urna interdum tortor sollicitudin iaculis. Aliquam purus est, venenatis ac blandit quis, semper quis felis. Integer arcu augue, accumsan at vulputate sed, tristique eu libero. Vivamus ut scelerisque massa. Pellentesque commodo arcu mollis dolor venenatis eleifend. Nulla sit amet rutrum nulla. Nullam leo ante, dapibus vel ipsum quis, bibendum condimentum ligula. Sed faucibus fermentum condimentum. Morbi eu ligula id lacus mattis pharetra. Phasellus auctor est sit amet sapien facilisis molestie vel in ipsum. Etiam malesuada vitae eros sed lacinia. Suspendisse eget iaculis odio, a molestie ex. Mauris ultrices et dolor nec dictum \cite{tseng_dis_spin}.

Vivamus elementum vehicula orci id mollis. Duis auctor sapien vel pretium bibendum. Nam aliquam, felis at efficitur pretium, justo libero cursus nisi, sit amet molestie metus massa nec sapien. Donec efficitur porttitor arcu et tempus. Phasellus pretium, diam id suscipit ultrices, lectus odio suscipit risus, at fermentum leo massa vitae eros. Donec elit orci, faucibus et aliquet quis, interdum eget lorem. Nulla a tincidunt odio, vitae commodo metus. Pellentesque bibendum cursus lacinia.

\begin{marginfigure}[{-10cm}]
 	\includegraphics[width=\textwidth]{chap1/spin}               
 	 \caption{Check it out, it's a Spin margin figure \url{spin.media.mit.edu}}
  	\label{fig:spin_margin}
\end{marginfigure}

\section{Section 1}

\begin{figure}[htb]
 	\includegraphics[width=\textwidth]{chap1/spin}               
 	 \caption{Check it out, it's a Spin \url{spin.media.mit.edu}}
  	\label{fig:spin}
\end{figure}

Phasellus eu nunc eget ante hendrerit porta. Etiam dignissim, mauris vitae luctus sollicitudin, metus purus iaculis tortor, eu lobortis arcu neque vitae ante. Donec egestas nec sem id vulputate. Ut efficitur non massa eget tempor. Nullam rhoncus odio sed dui fringilla semper. Nullam luctus odio felis, ac rutrum mauris maximus sodales. Phasellus non gravida nulla. Aenean congue sapien vitae facilisis luctus. Cum sociis natoque penatibus et magnis dis parturient montes, nascetur ridiculus mus. In laoreet ultrices tellus sed tincidunt. Aenean tempus, dui vel fermentum laoreet, sapien sapien facilisis turpis, vel volutpat sapien libero at mi. Maecenas eleifend libero in enim finibus, eu hendrerit ipsum ornare. Nulla placerat massa eget sapien tincidunt, non venenatis libero accumsan. Nunc ex lectus, rutrum sed varius sed, consectetur vel nisl. Aliquam eu eros vel metus sodales fermentum. Sed quis ultrices nisl, vel semper nibh (Table \ref{tab:sample_table}).

\begin{table}
  \centering
  \begin{tabular}{l l l l l}
    Column A & Column B & Column C & Column D & Column E \\
    \toprule
    A & B & C & D & E
  \end{tabular}
  \caption{A meaningless table}
  \label{tab:sample_table}
\end{table}

\appendix
\chapter{Appendix A}

\clearpage
\newpage

%% This defines the bibliography file (main.bib) and the bibliography style.
%% If you want to create a bibliography file by hand, change the contents of
%% this file to a `thebibliography' environment.  For more information 
%% see section 4.3 of the LaTeX manual.
\begin{singlespace}
\bibliography{main}
\bibliographystyle{apalike}
\end{singlespace}

\end{document}

